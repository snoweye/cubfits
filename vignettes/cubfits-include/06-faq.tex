
\section[FAQ]{FAQ}
\label{sec:faq}
\addcontentsline{toc}{section}{\thesection. FAQ}

\begin{enumerate}

\item {\bf\code{blue} Q:}
      What is the main difference of \code{cubfits()}, \code{cubappr()}, and
      \code{cubpred()}? \\
      {\bf\color{blue} A:}
      \code{cubfits()} is the usual MCMC method to estimate parameters where
      some people call backward simulation.
      On the other hand, \code{cubappr()} is pure simulation without
      observations from equilibrium states (if they have been reached.)
      \code{cubpred()} is more like machine learning techniques, such as
      cross-validation methods, use part of observations to estimate
      parameters, and predict the other parts of data.

\item {\bf\color{blue} Q:}
      Should \code{phi.Obs} be scaled? \\
      {\bf\color{blue} A:}
      For \code{cubappr()}, it must be scaled to mean 1 since the function uses
      as \code{phi.Obs} as initial values of $\phi_g$.
      For \code{cubfits()} and \code{cubpred()}, it is not necessary. The
      bias terms should be also estimated if \code{phi.Obs} were not
      scaled to mean 1. Post scaling of each MCMC iteration should be
      performed on parameters and prediction of $\phi_g$ to make them
      comparable with \code{cubappr()} results. However, this may induce
      bias as well.

\item {\bf\color{blue} Q:}
      What are configurations to enforce estimating bias of $\phi_g$,
      $K_{bias}$? \\
      {\bf\color{blue} A:}
      The only options currently \pkg{cubfits} have are in next.
\begin{Code}[title=Configuration]
  .CF.CT$type.p <- "lognormal_bias"
  .CF.CT$scale.phi.Obs <- FALSE
  .CF.CONF$estimate.bias.Phi <- TRUE
\end{Code}

\end{enumerate}

