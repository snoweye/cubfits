
\section[FAQs]{FAQs}
\label{sec:faqs}
\addcontentsline{toc}{section}{\thesection. FAQs}

\begin{enumerate}

\item {\bf\color{blue} Q:}
      Should the amino acids and ORFs be sorted for \pkg{cubfits}? \\
      {\bf\color{blue} A:}
      Yes. For performance issue, amino acids are coded with one character
      code, and ORFs and their expression data (if any) should be named.
      Both should be sorted by their names.

\item {\bf\color{blue} Q:}
      What is the main difference of \code{cubfits()}, \code{cubappr()}, and
      \code{cubpred()}? \\
      {\bf\color{blue} A:}
      \code{cubfits()} is the usual MCMC method to estimate parameters where
      some people call backward simulation.
      On the other hand, \code{cubappr()} is pure simulation without
      observations from equilibrium states (if they have been reached.)
      \code{cubpred()} is more like machine learning techniques, such as
      cross-validation methods, use part of observations to estimate
      parameters, and predict the other parts of data.

\item {\bf\color{blue} Q:}
      Should \code{phi.Obs} be scaled? \\
      {\bf\color{blue} A:}
      For \code{cubappr()}, it must be scaled to mean 1 since the function uses
      as \code{phi.Obs} as initial values of $\phi_g$.
      For \code{cubfits()} and \code{cubpred()}, it is not necessary. The
      bias terms should be also estimated if \code{phi.Obs} were not
      scaled to mean 1. Post scaling of each MCMC iteration should be
      performed on parameters and prediction of $\phi_g$ to make them
      comparable with \code{cubappr()} results. However, this may induce
      bias as well.

\item {\bf\color{blue} Q:}
      What is a better way to specify arguments of the major functions,
      \code{cubfits()}, \code{cubappr()}, or \code{cubpred},
      through arguments or configurations?\\
      {\bf\color{blue} A:}
      In general, \pkg{cubfits} strongly suggests configurations instead of
      arguments except for initial values of parameters. Default arguments of
      the major functions are limited and do not cover many useful or tedious
      options. As the development of options is stable, their should be removed
      from arguments and only changeable from configurations. Currently,
      the main objects for configurations are \code{.CF.CT} and \code{.CF.CONF}.

\item {\bf\color{blue} Q:}
      What are configurations to enforce estimating bias of $\phi_g$,
      $K_{bias}$? \\
      {\bf\color{blue} A:}
      The only options currently \pkg{cubfits} have are in next.
\begin{Code}[title=Configuration]
  .CF.CT$type.p <- "lognormal_bias"
  .CF.CT$scale.phi.Obs <- FALSE
  .CF.CONF$estimate.bias.Phi <- TRUE
\end{Code}

\item {\bf\color{blue} Q:}
      What is the default environment of \pkg{cubfits}? \\
      {\bf\color{blue} A:}
      \code{.cubfitsEnv} will dynamicly store generic functions, while
      all data are still located in \code{.GlobalEnv}.
      See Section~\ref{sec:misc} for examples.

\item {\bf\color{blue} Q:}
      What are the ways to access \pkg{cubfits} functions? \\
      {\bf\color{blue} A:}
      There are at least three levels of functions developed in \pkg{cubfits}:
      \begin{itemize}
        \item \code{cubfits::function\_name()} or simply \code{function\_name()}
              (if \pkg{cubfits} is loaded) can access exported major functions
              of \pkg{cubfits},
        \item \code{cubfits:::function\_name()}
              can access unexported internal functions and objects
              of \pkg{cubfits} (even \pkg{cubfits} is not loaded), and
        \item \code{.cubfitsEnv$function\_name()}
              can access generic functions dispatched in the environment
              \code{.cubfitsEnv} if \code{init\_function()} has been called.
              (\code{ls(.cubfitsEnv)} can see what have been dispatched.)
      \end{itemize}

\item {\bf\color{blue} Q:}
      What is the way to debug \pkg{cubfits} functions? \\
      {\bf\color{blue} A:}
      Debugger of negative \proglang{R} may not work well in some cases, so we
      suggest:
      \begin{itemize}
        \item \code{cat()} or \code{print()} and \code{source()}
              can be useful for non-generic functions, and
        \item \code{browser()} would be better for generic functions
              in \code{.cubfitsEnv}.
      \end{itemize}
      Note that if you \code{source()} a new code into \proglang{R}, you
      may still have to overwrite functions in \code{.cubfitEnv} since
      MCMC iterations heavily access functions from there instead of
      \code{.GlobalEnv}.

\item {\bf\color{blue} Q:}
      How does \pkg{cubfits} specify the reference codon for a amino acid? \\
      {\bf\color{blue} A:}
      \pkg{cubfits} follows \pkg{VGAM}'s arrangement to assign reference codon,
      default is the last one by names.
      For example, the amino acide ``Ala/A''
      has four synonymous codons, ``GCA'', ``GCC'', ``GCG'', and ``GCT'' where
      ``GCT'' will be the reference or base line of model.

\item {\bf\color{blue} Q:}
      How does \pkg{cubfits} arrange parameters
      $(\log(\vec{\mu}), \vec{\Delta t})$
      for codons and amino acids? \\
      {\bf\color{blue} A:}
      For ROC model, parameters (\code{b.Mat} in \code{bVec} format) are
      ordered by names of amino
      acid first, followed by $\mu$ for all non-reference codons, then
      $\Delta t$ for all non-ference codons. All non-reference codons are
      ordered by their name as well.
      For example, the amino acide ``Ala/A'' will have a parameter vector
      ordered as
      $\log(\mu_{GCA}), \log(\mu_{GCC}), \log(\mu_{GCG}),\allowbreak
       \Delta t_{GCA}, \Delta t_{GCC}, \Delta t_{GCG}$.
      See \code{help("bVec")} for more details.

\end{enumerate}

